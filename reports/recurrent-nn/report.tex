\documentclass[11pt, a4paper]{article}
\usepackage[utf8]{inputenc}
\usepackage[left=2cm, right=2cm, top=2.5cm, bottom=2.0cm]{geometry}
\usepackage{amsmath, amssymb, amsthm}
\usepackage[english]{babel}
\usepackage{graphicx}
\usepackage[font={small,it}]{caption}
\graphicspath{ {figures/} }
\usepackage{url}
\usepackage{appendix}
\usepackage{float}
\usepackage{multirow}
\usepackage[bottom]{footmisc}
\usepackage{wrapfig}
\usepackage{subcaption}
\usepackage{titling}
\setlength{\droptitle}{-10em}  

\title{ \huge Artificial neural networks \\ 
  { \large Assigment 3: Recurrent neural networks }}
\author{
        Lood, Cédric \\
        \small Master of Bioinformatics
}

\begin{document}
\maketitle

\section{Context}
In this exercise, we explore the use of recurrent neural networks and
their applications as associative memories. Items can be stored as
equilibrium points of the network, similarly to those in dynamical
systems.

\section{Hopfield network}

In this section, we use hopfield neural networks to store 10 digitized
handwritten digits. The network has an architecture of 240 neurons. It
is trained in such a way that each digit is an attractor point of the
network, where they represent minimums of the energy function. To note
is the architecture consisting of 240 neurons, so the storage of 10
patterns is possible with a small chance of error.

The first column of the left-hand side of figure \ref{fig:ndigits1}
corresponds to the patterns that are stored in the network. Each digit
is encoded as a 15 by 16 vector of pixels. Each 3 pairs of columns
following the digits in the first column corresponds to an attempt at
retrieving a noisy digit (the first column in a pair) through a fixed
amount of iterations. As can be seen, the process is fairly good,
failing only when the amount of noise is large - examples of failures
are located in the last column of the digits 2, incorrectly
reconstructed as a 7, and the digit 9, incorrectly reconstructed as a
2. In both cases, the error do seem to make sense to the human eye.

Similarly, the first column of the figure \ref{fig:ndigits2} displays
the original pattern. A fixed amount of noise is then introduced (2.5
in our example), then an iterative reconstruction process is
attempted.

\begin{figure}[H]
    \centering
    \begin{subfigure}{.5\textwidth}
      \centering
      \includegraphics[width=0.90\linewidth]{ndigits_fi1.png}
      \caption{Fixed amount of iterations (50)}
      \label{fig:ndigits1}
    \end{subfigure}%
    \begin{subfigure}{.5\textwidth}
      \centering
      \includegraphics[width=0.85\linewidth]{ndigits_fn2.png}
      \caption{Fixed amount of noise (2.5)}
      \label{fig:ndigits2}
    \end{subfigure}
    \caption{Hopfield network reconstruction of noisy digits}
    \label{fig:ndigit}
\end{figure}

\section{Elman network}



\bibliographystyle{ieeetr} 
\bibliography{bib-db}


\end{document}
